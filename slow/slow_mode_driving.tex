
\documentclass[12pt]{article}
%\usepackage{geometry} % see geometry.pdf on how to lay out the page. There's lots.
%\geometry{a4paper} % or letter or a5paper or ... etc
% \geometry{landscape} % rotated page geometry
\usepackage{amsmath, aas_macros}

% See the ``Article customise'' template for come common customisations

\title{Laplace-Fourier analysis for slow mode driving}
\author{Yuguang Tong}
%\date{} % delete this line to display the current date

%%% BEGIN DOCUMENT
\begin{document}

\maketitle

\section{Setup}
We drive slow mode in gyrokinetic system by adding $\delta B_{\parallel a}$ to Maxwell's equations. We begin with the Fourier transformed gyrokinetic equation:
%
\begin{equation}
\frac{\partial g_{\mathbf{k}s}}{\partial t} + i k_\parallel v_\parallel g_{\mathbf{k}s} + \frac{q_s}{T_s} v_\parallel F_{0s} i k_\parallel\tilde{\phi} = -\frac{q_s}{T_s} F_{0s} \frac{\tilde{A}}{\partial{t}}
\label{eq:gk_eq}
\end{equation}
%
where $\tilde{\phi}$ and $\tilde{A}$ are the source terms:
%
\begin{eqnarray}
\tilde{\phi} & = & j_0(k_\perp \rho_{\perp s})\phi_\mathbf{k}
 +  \frac{J_1(k_\perp \rho_{\perp s})}{k_\perp \rho_{\perp s}}\frac{mv_\perp^2}{q_s} 
\frac{\delta B_{\parallel\mathbf{k}}}{B_0} \\
\tilde{A} & =& J_0\left( k_\perp \rho_{\perp s}\right) \frac{v_\parallel A_{\parallel\mathbf{k}}}{c}
\end{eqnarray}
%
where $\rho_{\perp s} = v_{\perp}/\Omega_{cs}$. Note that $\phi$, $A_\parallel$ and $\delta B_\parallel$, $g_s$ are short for the Fourier components $\phi_\mathbf{k}$, $A_{\parallel \mathbf{k}}$, $\delta B_{\parallel \mathbf{k}}$ and $g_{s\mathbf{k}}$.

%%
The Poisson equation:
\begin{equation}
\begin{split}
\sum_s \frac{q_s^2 n_s}{T_s} (1 - \Gamma_{0s}(\alpha_s)) & \phi_\mathbf{k} - \sum_s q_s n_s \Gamma_{1s} (\alpha_s) \left( \frac{\delta B_{\parallel \mathbf{k}}}{B_0} + \frac{\delta B_{\parallel \mathbf{k}a}}{B_0}\right)\\
&\quad  = \sum_s q_s \int d^3 \mathbf{v} J_0(k_\perp \rho_{\perp s}) g_{\mathbf{k}s}
\end{split}
\end{equation}

%%
The parallel component of Ampere's law
\begin{equation}
\frac{ck_\perp^2}{4\pi} A_{\parallel\mathbf{k}} = \sum_s q_s \int d^3\mathbf{v} v_\parallel J_0(k_\perp \rho_{\perp s}) g_{\mathbf{k}s}
\end{equation}
where $\alpha_s = k_\perp^2 \rho_s^2/2$ 

%%
The perpendicular component of Ampere's law
\begin{equation}
\begin{split}
\sum_s q_s n_s \Gamma_{1s}(\alpha_s) \phi_\mathbf{k} + \left( \frac{\delta B_{\parallel \mathbf{k}}}{B_0} + \frac{\delta B_{\parallel \mathbf{k}a}}{B_0}\right) \left( \frac{B_0^2}{4\pi} + \sum_s n_sT_s\Gamma_{2s}(\alpha_s)\right) = \\
-\sum_s T_s \int d^3\mathbf{v} \frac{v_\perp^2}{v_{ts}^2}\frac{2J_1(k_\perp \rho_{\perp s})}{k_\perp \rho_{\perp s}} g_{\mathbf{k}s}
\end{split}
\end{equation}
%%%% section

\section{Laplace-Fourier solution with $A_\parallel=\phi = 0$}
\label{sec:apar=phi=0}

Consider antenna driving term
\begin{equation}
\delta B_{\parallel a} = \delta B_{\parallel 0} e^{i(\mathbf{k}_0\cdot \mathbf{r} - \omega_0 t)}
\label{eq:driving_b}
\end{equation}
Its Laplace transform is
\begin{equation}
\delta\hat{B}_{\parallel \mathbf{k} a} = \int_0^\infty  \delta B_{\parallel \mathbf{k}_0} e^{-i\omega_0 t} e^{-pt} dt = \frac{ \delta B_{\parallel \mathbf{k}_0}}{p+i\omega_0}
\label{eq:driving_b_LT}
\end{equation}

%%
Performing Laplace transform to the gyrokinetic equation gives
%
\begin{equation}
p\hat{g}_{\mathbf{k}s} -g_{\mathbf{k}s}(t=0) + ik_\parallel v_\parallel \hat{g}_{\mathbf{k}s} + ik_\parallel v_\parallel \frac{mv_\perp^2 F_{0s}}{T_s B_0} \frac{J_1(k_\perp \rho_\perp s)}{k_\perp \rho_{\perp s}} \frac{\delta \hat{B}_{\parallel \mathbf{k}}}{B_0} = 0
\end{equation}
%
Choosing zero initial condition, i.e., $g_{\mathbf{k}s}(t=0)=0$, the distribution function is solved to be:
%
\begin{equation}
\hat{g}_{\mathbf{k}s} =- \frac{ik_\parallel v_\parallel}{p + ik_\parallel v_\parallel} \frac{mv_\perp^2 F_{0s}}{T_s B_0} \frac{J_1(k_\perp \rho_{\perp s})}{k_\perp \rho_{\perp s}} \delta \hat{B}_{\parallel \mathbf{k}} 
\label{eqn:gks}
\end{equation}

%%
With $A_\parallel = \phi=0$, the perpendicular Ampere's law takes the form
\begin{equation}
\begin{split}
 \left( \frac{\delta B_{\parallel \mathbf{k}}}{B_0} + \frac{\delta B_{\parallel \mathbf{k}a}}{B_0}\right) \left( \frac{B_0^2}{4\pi} + \sum_s n_sT_s\Gamma_{2s}(\alpha_s)\right) = \\
 -\sum_s T_s \int d^3\mathbf{v} \frac{v_\perp^2}{v_{ts}^2}\frac{2J_1(k_\perp \rho_{\perp s})}{k_\perp \rho_{\perp s}} g_{\mathbf{k}s}
 \end{split}
 \label{eq:perp_ampere_1}
\end{equation}

Substitute Eq. (\ref{eqn:gks}) into the above equation and perform the integral:
\begin{equation}
\begin{split}
\mathrm{RHS} = \sum_s \frac{T_s n_s}{\pi^{3/2} v_{ts}^3} \frac{\delta B_{\parallel \mathbf{k}}}{B_0} \\ 
\quad \int 2\pi v_\perp dv_\perp \frac{4v_\perp^2}{v_{ts}^4} \frac{J_1^2(k_\perp \rho_{\perp s})}{(k_\perp \rho_{\perp s})^2} e^{-v_\perp^2/v_{ts}^2}\\
\quad \int dv_\parallel \frac{ik_\parallel v_\parallel}{p + ik_\parallel v_\parallel}e^{-v_\parallel^2/v_{ts}^2}
\end{split}
\end{equation}
%
Using relations A.1 and A.27 in AstroGK Manual (Howes et al. 2007), the integrals are expressed in terms of Bessel functions and plasma dispersion function:
\begin{equation}
\mathrm{RHS} = \sum_s T_s n_s \Gamma_{2s}(\alpha_s) \left( 1+\xi_s Z(\xi_s)\right)\frac{\delta \hat{B}_{\parallel \mathbf{k}}}{B_0}
\end{equation}
where $\xi_s = \frac{p}{-ikv_{ts}}$ and $Z(\xi) = \frac{1}{\sqrt{\pi}} \int_{-\infty}^\infty dt \frac{e^{-t^2}}{t-\xi}$ is the Plasma Dispersion Function. 
Equating LHS and RHS of Eq. (\ref{eq:perp_ampere_1}), we obtain that

\begin{eqnarray}
\delta \hat{B}_{\parallel \mathbf{k}} &=& \frac{ \frac{B_0^2}{4\pi} + \sum_s n_sT_s\Gamma_{2s}(\alpha_s)}{ -\frac{B_0^2}{4\pi} + \sum_s n_sT_s\Gamma_{2s}(\alpha_s) \xi_s Z(\xi_s)}\frac{ \delta B_{\parallel \mathbf{k}_0}}{p+i\omega_0} \\
&=&\frac{ \frac{1}{\beta_i} + \sum_s \frac{T_s}{T_i}\Gamma_{1s}(\alpha_s)}{ -\frac{1}{\beta_i} + \sum_s \frac{T_s}{T_i}\Gamma_{1s}(\alpha_s) \xi_s Z(\xi_s)}\frac{ \delta B_{\parallel \mathbf{k}_0}}{p+i\omega_0}
\end{eqnarray}
Note that  
%
\begin{equation}
D(p) = -\frac{1}{\beta_i} + \sum_s \frac{T_s}{T_i}\Gamma_{1s}(\alpha_s) \xi_s Z(\xi_s)
\end{equation} 
is exactly the dispersion relation, and is the same as Eq. (55) in \cite{Howes:2006a}. We know that there $D(p)=0$ has one  solution $p_1 = -i\omega_1 = - \gamma_1$ , corresponding to a non-propagating slow/ entropy mode. To obtain $\delta B_{\parallel \mathbf{k}}$, we apply the inverse Laplace transform via Bromwich integral:

\begin{eqnarray}
\delta B_{\parallel \mathbf{k}}(t) &=& \frac{1}{2\pi i} \int_{\beta - i\infty}^{\beta + i\infty} \frac{C_b e^{pt} dp}{D(p) (p-p0)} \\
&=& \mathrm{Res}(p_1) + \mathrm{Res}(p_0)
\end{eqnarray}
where 
\begin{equation}
C_b =  \left[\frac{1}{\beta_i} + \sum_s \frac{T_s}{T_i}\Gamma_{1s}(\alpha_s)\right] \delta B_{\parallel \mathbf{k}_0}
\end{equation}
 is a frequency-independent term. $p0=-i\omega_0$. $\mathrm{Res}(p)$ denotes residue at simple pole $p$. 

The Residues are evaluated as follows:
%
\begin{equation}
\mathrm{Res}(p_0) = \frac{C_b e^{p_0t}}{D(p_0)}  = \frac{C_b e^{-i\overline{\omega}_0\overline{t}}}{D(p_0)}
\end{equation}
where $\overline{\omega}_ 0 = \frac{p_0}{-i k_\parallel v_A} = \frac{\omega_0}{k_\parallel v_A}$ and $\overline{t} = t k_\parallel v_A$.
%
And 
\begin{eqnarray}
\mathrm{Res}(p_1) &=& \left[ \frac{(p-p_1) C_b e^{pt}}{D(p) (p-p_0)}\right]_{p=p_1} \\
&=& \frac{C_b e^{p_1t}}{\left[\frac{dD}{dp}\right]_{p_1} (p_1-p_0)}
\end{eqnarray}

Define
\begin{eqnarray}
G(p) &=& \frac{dD}{dp} (p-p_0)\\
&=& \sum_s \Gamma_{1s} \frac{T_s}{T_i} \frac{1}{\beta_i} \sqrt{\frac{T_s}{T_i}\frac{m_s}{m_i}} \left[(1- 2\xi_s^2)Z_s - 2\xi_s\right] (\overline{\omega} - \overline{\omega}_0)
\end{eqnarray}

Hence 
\begin{equation}
\mathrm{Res}(p_1) = \frac{C_b e^{-i \overline{\omega}_1 \overline{t}}}{G(\overline{\omega}_1)}
\end{equation}

To summarize, parallel magnetic field fluctuation is given by
\begin{equation}
\delta B_{\parallel \mathbf{k}}(t) = \frac{C_b e^{-i\overline{\omega}_0\overline{t}}}{D(p_0)} +  \frac{C_b e^{-i \overline{\omega}_1\overline{t}}}{G(\overline{\omega}_1)} 
\end{equation}

\section{Laplace-Fourier solution: general case}
We take away the constraint on $\delta A_{\mathbf{k}\parallel}$ and $\phi$. The process is the similar to the restricted case we considered in the above section. We obtain an expression for the Fourier-Laplace-transformed distribution function $\hat{g}_{\mathbf{k}s}(p)$ from the gyrokinetic equation. Substituting it into the three Maxwell's equations (Poisson, parallel and perpendicular components of Ampere's law), we obtain three equations for three independent field fluctuations, i.e., $\delta \phi$, $\delta A_\parallel$ and $\delta B_\parallel$. Doing inverse Laplace transform will give us the temporal evolution for the fields.

Apply Laplace transform to Eq. (\ref{eq:gk_eq}) and set initial conditions to zero gives:
%
\begin{equation}
\hat{g}_{\mathbf{k}s}= - \frac{q_sF_{0s}}{T_s} \frac{ik_\parallel v_\parallel}{p+ik_\parallel v_\parallel} \left(
J_{0s} \hat{\phi}_{\mathbf{k}} + \frac{J_{1s}}{k_\perp \rho_{\perp s}} \frac{mv_\perp^2}{q_s} \frac{\delta \hat{B}_{\parallel \mathbf{k}}}
{B_0} + J_{0s}\frac{p \hat{A}_{\parallel \mathbf{k}}}{ik_\parallel c}\right)
\label{eq:lt_gk_general}
\end{equation}

Substituting Eq. (\ref{eq:lt_gk_general}) into Poisson's equation and performing integration over velocity yield
%
\begin{equation}
\begin{split}
\sum_s \frac{q_s^2 n_s}{T_s} \left[ ( 1 + \Gamma_{0s}\xi_s Z_s)\left(\hat{\phi}_{\mathbf{k}} - \frac{ip\hat{A}_{\parallel\mathbf{k}}}{k_\parallel c}\right) + (1-\Gamma_{0s}) \frac{ip\hat{A}_{\parallel\mathbf{k}}}{k_\parallel c}\right] \\
 + \sum_s q_s n_s \Gamma_{1s}\left( \xi_s Z_s \frac{\delta \hat{B}_{\parallel \mathbf{k}}}{B_0} - \frac{\delta \hat{B}_{\parallel \mathbf{k} a}}{B_0}\right) = 0
\end{split}
\end{equation}
For convenience, define the following dimensionless quantities:
%
\begin{eqnarray}
A &=& \sum_s \frac{T_i}{T_s} (1 + \Gamma_{0s}\xi_s Z_s) \\
B &=& \sum_s \frac{T_i}{T_s} (1 - \Gamma_{0s}) \\
C &=& \sum_s \frac{q_i}{q_s} \Gamma_{1s} \xi_s Z_s \\
D &=& \sum_s \frac{2T_s}{T_i}\Gamma_{1s} \xi_s Z_s \\
E &=& \sum_s \frac{q_i}{q_s} \Gamma_{1s} \\
F &=& \sum_s \frac{2T_s}{T_i} \Gamma_{1s}
\end{eqnarray}
%
and
\begin{eqnarray}
X &=& \hat{\phi}_{\mathbf{k}} - \frac{ip\hat{A}_{\parallel \mathbf{k}}}{k_\parallel c} = \hat{\phi}_{\mathbf{k}} - \frac{\omega\hat{A}_{\parallel \mathbf{k}}}{k_\parallel c} \\
Y &=& \frac{ip\hat{A}_{\parallel \mathbf{k}}}{k_\parallel c} = \frac{\omega\hat{A}_{\parallel \mathbf{k}}}{k_\parallel c} \\
Z &=& \frac{T_i}{q_i} \frac{\delta \hat{B}_{\parallel \mathbf{k}}}{B_0} \\
Z_a &=& \frac{T_i}{q_i} \frac{\delta \hat{B}_{\parallel \mathbf{k}a}}{B_0} \\
\overline{\omega} &=& \frac{\omega}{k_\parallel v_A} = \frac{p}{-ik_\parallel v_A}
\end{eqnarray}

Assuming hydrogen plasma: $q_i = -q_e$, $n_{0i}=n_{0e}$, and using the above definition, the Poisson equation becomes:
%
\begin{equation}
A X + B Y + CZ = E Z_a
\end{equation}
Similarly, the parallel component of the Ampere's law yields
%
\begin{equation}
(A-B) X + \frac{\alpha_i}{\overline{\omega}^2} Y + (C+E) Z = 0
\end{equation}
The perpendicular component of the Ampere's law yields: 
%
\begin{equation}
C X - E Y + \left(D-\frac{2}{\beta_i}\right) Z = \left(F + \frac{2}{\beta_i}\right) Z_a
\end{equation}

Combining the above three equations in the matrix form gives
\begin{equation}
\begin{pmatrix}
A & B & C \\
A-B & \alpha_i/\overline{\omega}^2 & C + E \\
C & -E & D- 2/\beta_i 
\end{pmatrix}
\begin{pmatrix}
 X\\
 Y\\
 Z
\end{pmatrix}
=
\begin{pmatrix}
EZ_a \\
0\\
(F + 2/\beta_i) Z_a
\end{pmatrix}
\label{eq:general_disp_equation}
\end{equation}

Notice that the $3\times 3$  matrix in the above expression is exactly the dispersion tensor for linear collisionless gyrokinetics with spatial homogeneity (c.f. Eq. (6.24) in AstroGK manual (Howes, 2007) and Eq. (C15) in \cite{Howes:2006a}).

\section{Driving gyrokinetics with $A_\parallel = 0$}
To simplify comparison with AstroGK simulation of linearly driving slow mode, we set $A_\parallel=0$. We choose the Poisson's equation and the perpendicular component of the Ampere's law to solve for fields:
%
\begin{equation}
\begin{pmatrix}
A & C\\
C & D-2/\beta_i 
\end{pmatrix}
\begin{pmatrix}
X \\
Z
\end{pmatrix}
=
\begin{pmatrix}
E \\
F + 2/\beta_i
\end{pmatrix}
Z_a
\end{equation}

The Fourier-Laplace transformed fields $X$ and $Z$ are obtained by multiplying the matrix inverse:
%
\begin{equation}
\begin{pmatrix}
\hat{\phi} (p)\\
\frac{T_i}{q_i} \frac{\delta \hat{B}_{\parallel \mathbf{k}}(p)}{B_0} 
\end{pmatrix}
=
\begin{pmatrix}
X \\
Z
\end{pmatrix}
=\frac{1}{\mathrm{Det}M}
\begin{pmatrix}
D-2/\beta_i  & -C\\
-C & A
\end{pmatrix}
\begin{pmatrix}
E \\
F + 2/\beta_i
\end{pmatrix}
Z_a
\end{equation}
Using the same antenna driving term $\delta B_{\parallel a}$ as in Eq. (\ref{eq:driving_b}) and its Laplace transform Eq. (\ref{eq:driving_b_LT}), we obtain

\begin{eqnarray}
\begin{pmatrix}
\phi (t)\\
\frac{T_i}{q_i} \frac{\delta B_{\parallel \mathbf{k}}(t)}{B_0} 
\end{pmatrix}
&= &\nonumber \\
 & & ILT\left[
\frac{1}{\mathrm{Det}M}
\begin{pmatrix}
D-2/\beta_i  & -C\\
-C & A
\end{pmatrix}
\begin{pmatrix}
E \\
F + 2/\beta_i
\end{pmatrix}
 \frac{T_i}{q_i} \frac{\delta B_{\parallel \mathbf{k}0}}{B_0}
 \frac{1}{p-p_0}
\right] \nonumber \\
&=&\sum_{i=0}^n \mathrm{Res}(p_i)
\end{eqnarray}
where the sum is over all the simple poles $p_i$. In particular, $p_0 = -i\omega_0$ results from antenna driving. 

To evaluate residues at $p_i$ for $i\neq 0$, we expand Det$M$ in Taylor series near $p_i$ in the same way as we did in Section \ref{sec:apar=phi=0}:
%
\begin{equation}
\mathrm{Res}(p_i) = 
\left[
\frac{1}{\frac{d\mathrm{Det}M}{dp}}
\begin{pmatrix}
D-2/\beta_i  & -C\\
-C & A
\end{pmatrix}
\begin{pmatrix}
E \\
F + 2/\beta_i
\end{pmatrix}
 \frac{T_i}{q_i} \frac{\delta B_{\parallel \mathbf{k}0}}{B_0}
 \frac{e^{pt}}{p-p_0}
\right]_{p=p_i}
\end{equation}
where 
\begin{equation}
\frac{d\mathrm{Det}M}{dp} = A^\prime\left(D-\frac{2}{\beta_i}\right) + 
AD^\prime - 2CC^\prime 
\end{equation}
%
$A^\prime, C^\prime, D^\prime$ are derivatives w.r.t. $p$:
\begin{eqnarray}
A^\prime &=&\frac{dA}{dp} = \sum_s \frac{T_i}{T_s} \Gamma_{0s}G_s \\
C^\prime &=& \frac{dC}{dp} = \sum_s \frac{q_i}{q_s} \Gamma_{1s} G_s \\
D^\prime &=& \frac{dD}{dp} = \sum_s \frac{2T_s}{T_i} \Gamma_{1s} G_s 
\end{eqnarray}
where 
\begin{eqnarray}
G_s &=& \frac{\left(1-2\xi_s^2\right) Z_s - 2 \xi_s}{-ik_\parallel v_{ts}} \nonumber \\
&=& \frac{\left(1-2\xi_s^2\right) Z_s - 2 \xi_s}{-ik_\parallel v_{A}} \frac{1}{\sqrt{\beta_i}}
\sqrt{\frac{m_s}{m_i}\frac{T_i}{T_s}}
\end{eqnarray}


\bibliographystyle{plain}
\bibliography{ref}



\end{document}
